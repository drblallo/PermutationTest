\section{introduzione}


\subsection{Obiettivi}
Questo documento intende presentare un implementazione openCL del permutation test, ovvero un test di verifica di ipotesi,  con lo scopo di offrire velocità di computazione maggiore tramite l'uso schede grafiche, hardware specializzato per il calcolo parallelo. 
In particolare, i risultati raggiunti sono:
\begin{itemize}
	\item \textbf{Test multivariati} l'analisi di campioni multidimensionali risulta particolarmente difficile e costosa, la nostra implementazione risolve tale problema per costruzione.
	\item \textbf{Memoria lineare} il nostro algoritmo richiede memoria pari a quella necessaria per allocare l'input in ingresso. Tale soluzione è ideale poiché la memoria disponibile sulle schede grafiche è limitata e garantire tale proprietà nel problema affrontato è difficile.
	\item \textbf{Accesso sequenziale} il nostro algoritmo accede alla memoria in maniera sequenziale ed esattamente una volta per ogni elemento, tale soluzione è ideale poiché l'accesso non ordinato rallenta significativamente l'efficienza delle schede grafiche.
\end{itemize}


\subsection{Descrizione del problema}
Varie branche scientifiche e economiche moderne, come ad esempio la diagnostica medica, si basano sulla capacità di discernere le differenze tra popolazioni di oggetti. Se è possibile descrivere le popolazioni tramite variabili numeriche allora ne consegue il problema rientra nel campo della matematica.
Le procedure che permettono di operare tali distinzioni sono dette test di verifica di ipotesi e la loro formulazione rigorosa giace nel dominio della statistica.
Talvolta il costo computazionale necessario per risolvere tali operazioni è assai elevato, in particolare perché la complessità degli algoritmi è spesso legata alla dimensione del campione. Inoltre se si tenta di analizzare un evento reale che avviene ad alta frequenza è importante riuscire a garantire la maggiore velocità di analisi possibile, poiché ciò permette di estendere l'uso dei test di ipotesi a nuovi ambiti.
Per queste ragioni è interessante studiare implementazioni efficienti dei test statistici.

\paragraph{Permutation Test}
In alcune situazioni è possibile porre delle assunzioni riguardo quali sono i parametri delle popolazioni che si vuole studiare, in tali casi è spesso possibile creare dei test specifici particolarmente potenti. Tali test prendono il nome di test parametrici.
Altre volte le caratteristiche delle popolazioni sono completamente ignote, poiché magari derivano da eventi naturali. I test ideati per lavorare con queste popolazioni prendono il nome di test non parametrici. 
I test di permutazione sono particolari all'interno di questa categoria perché il loro funzionamento li rende adatti alla parallelizzazione, e di conseguenza candidati ideali ad essere computati su schede grafiche.


